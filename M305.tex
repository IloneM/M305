\documentclass{report}
\pagestyle{headings}
\usepackage{ucs}
\usepackage[utf8x]{inputenc}
\usepackage[hmargin=2.5cm, vmargin=1.5cm]{geometry}
\usepackage[T1]{fontenc}
\usepackage[francais]{babel}
\usepackage{titlesec} 
\usepackage{upgreek}
\usepackage{pifont}
\usepackage{graphicx} 
\usepackage{pstricks-add}
\usepackage{pst-eucl}
\usepackage{amsmath}
\usepackage{amssymb}
\usepackage{amsthm}
\usepackage{enumitem}
\usepackage{stmaryrd}




%________________________________________________________________________%

% Nouvelles commandes (raccourcis)

% Ensembles

\newcommand{\D}{\mathcal{D}}
\newcommand{\R}{\mathbb{R}}
\newcommand{\Q}{\mathbb{Q}}
\newcommand{\Z}{\mathbb{Z}}
\newcommand{\C}{\mathbb{C}}
\newcommand{\K}{\mathbb{K}}
\newcommand{\N}{\mathbb{N}}
\newcommand{\ent}[2]{[\![ #1, #2 ]\!]}



% Algèbre linéaire

\newcommand{\ev}{espace vectoriel}
\newcommand{\sev}{sous-espace vectoriel}
\newcommand{\kev}{$\K$-\ev}
\newcommand{\jus}{\underline{Justification :}}

\newcommand{\sta}{\mathrm{Stab}}
\newcommand{\card}[1]{\mathrm{card}(#1)}
\newcommand{\vect}[1]{\mathrm{Vect}\left(#1\right)}
\newcommand{\rg}[1]{\mathrm{rg}(#1)}
\newcommand{\mat}[2]{\mathrm{Mat}(#1 , #2)}

\newcommand{\im}{\mathop{\mathrm{Im}}\nolimits}
\renewcommand{\ker}{\mathop{\mathrm{Ker}}\nolimits}



% Fonctions hyperboliques

\renewcommand{\th}{\mathop{\mathrm{th}}\nolimits}
\newcommand{\ch}{\mathop{\mathrm{ch}}\nolimits}
\newcommand{\sh}{\mathop{\mathrm{sh}}\nolimits}

\newcommand{\argsh}{\mathop{\mathrm{Argsh}}\nolimits}
\newcommand{\argch}{\mathop{\mathrm{Argch}}\nolimits}
\newcommand{\argth}{\mathop{\mathrm{Argth}}\nolimits}



% Raccourcis


\newcommand{\ssi}{si et seulement si}

\newcommand{\equi}{les assertions suivantes sont équivalentes :}

\newcommand{\pgcd}{\mathrm{PGCD}} 

\newcommand{\con}{$\text{\ding{54}}$ Contradiction.} 

\newcommand{\dis}{\displaystyle}
\newcommand{\dx}{\mathrm{d} x}
\newcommand{\dt}{\mathrm{d} t}




% Topologie

\newcommand{\bfe}{\overline{B}}

\newcommand{\nor}{\Vert \ \Vert}
\newcommand{\norm}[1]{\left\Vert #1 \right\Vert}
%\newcommand{\nt}[1]{|\!|\!|  #1 |\!|\!| }%
\newcommand{\nt}[1]{\llbracket  #1 \rrbracket }

\newcommand{\ov}[1]{\overline{#1}}


\newcommand{\inte}[1]{\mathring{ #1}}
\newcommand{\ad}[1]{\overline{ #1}}
\newcommand{\fr}[1]{\text{Fr}(#1) }
\newcommand{\lims}[1]{\overset{n \rightarrow + \infty}{\underset{#1}{\longrightarrow}}}
%\newcommand{\limf}[2]{\overset{x \rightarrow #1}{\underset{\nor_{#2}}{\longrightarrow}}}%
\newcommand{\limsu}{\overset{n \rightarrow + \infty}{\longrightarrow}}
\newcommand{\limf}[1]{\overset{x \rightarrow #1}{\longrightarrow}}





\newcommand{\mn}[1]{\mathcal{M}_n(#1)}
\newcommand{\mnk}{\mn{\K}}
\newcommand{\gl}[1]{\text{GL}(#1)}
\newcommand{\gle}{\gl{E}}
\newcommand{\gln}[1]{\text{GL}_n(#1)}
\newcommand{\glnk}{\gln{\K}}
\newcommand{\on}[1]{\text{O}(#1)}
\newcommand{\onn}{\on{n}}
\newcommand{\son}[1]{\text{SO}(#1)}
\newcommand{\sonn}{\son{n}}
\renewcommand{\mat}[2]{\underset{#1}{\text{Mat}}#2}


\newcommand{\tra}[1]{\,{\vphantom{#1}}^{t}\!{#1}}

\newcommand{\lir}[2]{\substack{#1 \to #2}}
\newcommand{\li}[2]{\lim\limits_{\substack{#1 \to #2}}}
\newcommand{\lix}[1]{\lim\limits_{\substack{#1}}}
\newcommand{\lis}{\li{n}{+\infty}}

\newcommand{\un}{{(u_n)}_{n \in \N} }
\newcommand{\uu}[1]{{(u_{#1})}_{n \in \N}}
\newcommand{\pp}[1]{{\left( #1\right)}_{n \in \N}}

\newcommand{\fn}{{(f_n)}_{n \in \N} }
\newcommand{\sfn}{\dis \sum f_n }


\newcommand{\f}{\forall \ }
\newcommand{\e}{\exists \ }
\newcommand{\va}[1]{\left\vert #1 \right\vert }

\renewcommand{\o}[1]{\mathrm{o}\left(#1\right)}

     
%________________________________________________________________________%

% Théorème, définition ...

\theoremstyle{definition}   
\newtheorem*{defi}{Définition}
\newtheorem*{depro}{Définition-proposition}
\newtheorem*{theo}{Théorème} 
\newtheorem*{prop}{Proposition}
\newtheorem*{lem}{Lemme} 
\newtheorem*{coro}{Corollaire} 
\newtheorem*{ax}{Axiome}
                
\theoremstyle{remark}
\newtheorem*{rem}{Remarque}
\newtheorem*{nota}{Notation} 


\newcommand*{\quo}[2]%
{\ensuremath{%
    #1/\!\raisebox{-.65ex}{\ensuremath{\mathcal{#2}}}}}

%________________________________________________________________________%

% Définition personnalisée de la numerotation des titres, sous-titres ...

\renewcommand{\thesection}{\Roman{section}.}
\renewcommand{\thesubsection}{\alph{subsection}.}
\renewcommand{\thesubsubsection}{\alph{subsubsection})}

%________________________________________________________________________%

% Structure

% Titres personnalisés

\newcommand{\partie}[1]{\textcolor{red}{\section{#1}}}
\newcommand{\titre}[1]{\textcolor{orange}{\subsection{#1}}}

% Commande exercice et exemple

\newcommand{\exems}{{\textbf{$\text{\ding{46}}$ Exemples : \ }}}

\newcommand{\exem}{{\textbf{$\text{\ding{46}}$ Exemple : \ }}}

\newcounter{nexo}
\setcounter{nexo}{0}        
\newcommand{\exo}{
  \stepcounter{nexo}        
  {\textbf{Exercice \arabic{nexo}} \\ }}
  
\newcommand{\appli}{{\textbf{$\text{\ding{46}}$ Application : \ }}}  

% Application

\newcommand{\app}[5]{\begin{array}[t]{lrcl}
#1 : & #2 & \longrightarrow & #3 \\
& #4 & \longmapsto & #5  \\
\end{array}
}

% Encadrement

\newcommand{\enc}[1]{\noindent\fbox{\parbox{\linewidth\fboxrule\fboxsep}{#1}}}
  
%________________________________________________________________________%

% Style de numérotation équation
  
\renewcommand{\theequation}{\arabic{equation}}
  
%________________________________________________________________________%

% Listes théorèmes, définitions
  
\newlist{az}{enumerate}{3}
\setlist[az,1]{label=\fcolorbox{blue}{lightgray}{\alph*},font=\color{black}}
\setlist[az,2]{label=(\roman*)}
\setlist[az,3]{label=$\leadsto$}

%________________________________________________________________________%

% Listes théorèmes, définitions correspondant aux démonstrations  
  
\newlist{dem}{enumerate}{1}
\setlist[dem,1]{label=\textbf{\alph*.}}


%________________________________________________________________________%

% Listes exercices, exemples
  
\newlist{exe}{enumerate}{2}
\setlist[exe,1]{label=\textcircled{\arabic*} }
\setlist[exe,2]{label= \alph*)}

%________________________________________________________________________%

% Structure démonstration (existence et unicité)

\newcommand{\demo}[2]{\begin{enumerate}[label=\textbullet]
\item \textsc{Existence :} #1
\item \textsc{Unicité :} #2
\end{enumerate}}

%________________________________________________________________________%

% Liste avec point
  
\newlist{point}{enumerate}{1}
\setlist[point,1]{label=$\bullet$}

%________________________________________________________________________%

% Liste numérotée
    
\newlist{lnu}{enumerate}{1}
\setlist[lnu,1]{label=(\arabic*) }

% Liste avec étoile 

\newlist{etoile}{enumerate}{2}
\setlist[etoile,1]{label=$\star$}
\setlist[etoile,2]{label= -}

%________________________________________________________________________%



% Listes cas
  
\newenvironment{cas}[1][gray]
  {\begin{description}[font=\color{#1}]}
  {\end{description}}



%________________________________________________________________________%

%_______________________________________________________________________Début document_____________________________________________________________________%


 
\begin{document}

\chapter*{Algèbre 2}

\subsection{Groupes abéliens de type fini}

\begin{depro}
Soient $G$ un groupe abélien et $S$ une partie de $G$. Alors il existe un unique sous-groupe $H$ de $G$ qui contient $S$ minimal (pour l'inclusion). $H$ est appelé le \textit{sous-groupe de $G$ engendré par la partie $S$}, et noté $<S>$.
\end{depro}

\begin{proof}
On note $\mathcal{E}$ l'ensemble des sous-groupes de $G$ contenant $S$. $\mathcal{E}$ est non vide (car $G \in\mathcal{E}$). Alors $\dis \bigcap_{H \in \mathcal{E}} H$ est un sous-groupe de $G$ contenant $S$ qui contenu dans tout autre élément de $\mathcal{E}$.
\end{proof}

\begin{rem}
Soit $G$ un groupe abélien, soit $S$ une partie de $G$. Alors $<S>$ contient l'élément neutre, et stable par additivité et passage à l'inverse. Comme $G$ est abélien, $<S>$ contient donc toutes les combinaisons linéaires d'éléments de $S$ : $<S>=\left\{\dis \sum n_is_i \mid (s_i)_{1 \leqslant i \leqslant r} \in S^r, (n_i)_{1 \leqslant i \leqslant r} \in \Z^r, r \in \N \right\}$ (un tel ensemble est bien un sous-groupe de $G$ d'où l'égalité par unicité).
\end{rem}

\begin{defi}
Soit $G$ un groupe abélien, soit $S$ une partie de $G$. 
\begin{point}
\item On dit que $S$ est u\textit{ne partie génératrice} de $G$, ou encore que $S$ engendre $G$, si $G=<S>$.
\item On dit que le \textit{groupe abélien} $G$ est de \textit{type fini} s'il admet une \textit{partie génératrice finie}. 
\end{point}
\end{defi}

Pour tout $r \in \N^*$., on note $e=(e_i)_{1 \leqslant i \leqslant r} \in (\Z^r)^r$ la base canonique de $\Z^r$ (chaque vecteur a pour ième coordonnée $1$ et $0$ ailleurs).

\exem Un groupe $G$ abélien fini est de type fini (engendré par la partie $G$ finie). \\
Soit $r \in \N^*$. Alors $\Z^r$ est un groupe abélien de type fini (engendré par la partie $\{e_1, \ldots, e_r \}$ finie).

\begin{lem}
Soit $G$ un groupe abélien, soit $r \in \N^*$, soit $(x_1, \ldots, x_r) \in G^r$. Alors il \textit{existe un unique morphisme} de groupes $f$ du groupe $\Z^r$ dans le groupe $G$ qui, pour tout $i \in \nt{1,r}$, \textit{envoie $e_i$ sur $x_i$ $(\star)$}, donné par $\app{f}{\Z^r}{G}{(a_i)_{1 \leqslant i \leqslant r}}{\dis \sum_{i=1}^ra_ix_i}$. \\ De plus, $\im f =<\{x_1, \ldots, x_r\}>$.
\end{lem}

\begin{proof}
Un morphisme de groupes de $\Z^r$ dans $G$ vérifiant $(\star)$ est uniquement déterminé par l'image de la base canonique $e$ par décomposition d'un vecteur de $\Z^r$ dans cette base et linéarité du morphisme.
\end{proof}

\begin{coro}
Un groupe abélien est de type fini si et seulement si il existe un morphisme de groupes \textit{surjectif} du groupe $\Z^r$ dans le groupe $G$ (où $r \in \N$).
\end{coro}

\begin{proof}
On conserve les notations de la proposition précédente. \\
Soit $G$ un groupe abélien de type fini. Alors $G$ possède une partie génératrice finie $S=\{x_1, \ldots, x_r \}$ (où $r \in \N$), d'où $\im f=<S>=G$, ie le morphisme de groupes $f$ de $\Z^r$ dans $G$ est surjectif. \\
Réciproquement, soit $g$ un morphisme de groupes surjectif de $\Z^r$ (où $r \in \N$) dans $G$ , alors $S=\{f(e_1), \ldots, f(e_r) \}$ est une partie génératrice de $G$ (car tout élément de $G$ possède un antécédent dans $\Z^r$ qui se décompose dans la base canonique $e$, donc son image, ie $g$, s'écrit comme combinaison linéaire d'éléments de $S$). 
\end{proof}

\begin{prop}
Soient $G$ et $H$ deux groupes abéliens. Soit $f$ un morphisme de groupes de $G$ dans $H$. On suppose que le groupe abélien $G$ est de type fini. Alors le groupe abélien $\im f$ est de type fini.
\end{prop}

\begin{proof}
Soit $S$ une partie génératrice de $G$. Alors $f(S)$ est une partie génératrice de $\im f$.
\end{proof}

\begin{coro}

\end{coro}

\begin{prop}
Soient $G$ et $H$ deux groupes abéliens. Soit $f$ un morphisme de groupes de $G$ dans $H$. On suppose que les groupes abéliens $\im f$ et $\ker f$ sont de \textit{type fini}. Alors $G$ est un groupe abélien de type fini.
\end{prop}

\begin{proof}
Par hypothèse, il existe $(x_1, \ldots, x_r) \in G^r$ tel que la partie finie $\{f(x_1), \ldots, f(x_r) \}$ engendre $\im f$ et il existe $(y_1, \ldots, y_r) \in (\ker f)^s$ tel que la partie finie $\{y_1, \ldots, y_s \}$ engendre $\ker f$. Soit $g \in G$. Alors il existe $(n_1, \ldots, n_r) \in \Z^r$ tel que $f(g)=\dis \sum_{i=1}^r n_if(x_i)=f\left(\sum_{i=1}^r n_ix_i\right)$. On conclut en décomposant l'élément $g-\dis \sum_{i=1}^r n_ix_i\in \ker f$ selon la famille $\{y_1, \ldots, y_s\}$.
\end{proof}



\begin{prop}
Soit $G$ un groupe abélien de type fini. Soit $H$ un sous-groupe de $G$. Alors le groupe abélien $H$ est de type fini.
\end{prop}

\begin{proof} \
\begin{itemize}
\item On suppose le groupe $G$ monogène. Si $G$ est cyclique, on sait alors que $H$ est cyclique. Sinon, on se donne un générateur $g$ de $G$ (qui sont tous d'ordre infinis). On sait que l'application $\app{\varphi}{\Z}{G}{n}{ng}$ est un isomorphisme de groupes. $H$ est isomorphe au sous-groupe $\varphi^{-1}(H)$ de $\Z$ qui s'écrit $\varphi^{-1}(H)=m\Z$ (où $m \in \N$), ie $H=\varphi(m\Z)=\{ng \mid n \in m\Z \}=<mg>$ est monogène. 
\item Dans le cas contraire, on considère un morphisme de groupes $f$ surjectif de $\Z^r$ dans $G$, où $r \in \N^*$ désigne le cardinal de la partie génératrice de $G$ (cf preuve du corollaire précédent). Montrons par récurrence sur $r \in \N^*$ que $H$ est de type fini. \\
Le cas $r=1$, d'après la remarque précédente, a été traité dans le premier tiret. \\
On suppose le résultat vrai pour $r-1$ ($r \in \N^*$). On considère groupe $\Z^{r-1}$ comme sous-groupe de $\Z^r$ (via l'injection de $\Z^{r-1} \times \{0\}$ dans $\Z^{r}$). Par hypothèse de récurrence, tout sous-groupe de $K=f(\Z^{r-1})$ est de type fini. Montrons que $G/K$ est monogène : considérons le morphisme de groupes $\app{\psi}{\Z}{G/K}{a}{\overline{f(0, \ldots, 0,a)}}$. 
Soit $\overline{g} \in G/K$. Par surjectivité de $f$, il existe $(a_1, \ldots, a_r) \in \K^r$ tel que $g=f(a_1, \ldots, a_r)$. Donc $g=\underbrace{f(a_1, \ldots, a_{r-1},0)}_{\in K}+f(0, \ldots, 0, a_r)$, ie $\overline{g}=\overline{f(0, \ldots, 0,a_r)}=\psi(a_r)$. Ainsi, $\psi$ est surjectif. \\
On considère $\pi_H$ le morphisme de groupes canonique (surjectif) de projection de $H$ dans $G/K$. $\im \pi_H$ est un sous-groupe de $G/K$ monogène donc $\im \pi_H$ est monogène. \\
$\ker \pi_H=\K \cap H \subset K$ est de type fini par hypothèse de récurrence. D'où, d'après la proposition précédente, $H$ est de type fini.
\end{itemize}
\end{proof}

\end{document}



