\documentclass{article}
\usepackage[a4paper]{geometry}   % .. avec les bonnes marges
\pagestyle{headings}
\usepackage{ucs}
\usepackage[utf8x]{inputenc}
%\usepackage[hmargin=2.5cm, vmargin=1.5cm]{geometry}
\usepackage[T1]{fontenc}
\usepackage[francais]{babel}
\usepackage{titlesec} 
\usepackage{upgreek}
\usepackage{pifont}
\usepackage{graphicx} 
\usepackage{pstricks-add}
\usepackage{pst-eucl}
\usepackage{amsmath}
\usepackage{amssymb}
\usepackage{amsthm}
\usepackage{enumitem}
\usepackage{stmaryrd}




%________________________________________________________________________%

% Nouvelles commandes (raccourcis)

% Ensembles

\newcommand{\D}{\mathcal{D}}
\newcommand{\R}{\mathbb{R}}
\newcommand{\Q}{\mathbb{Q}}
\newcommand{\Z}{\mathbb{Z}}
\newcommand{\C}{\mathbb{C}}
\newcommand{\K}{\mathbb{K}}
\newcommand{\N}{\mathbb{N}}
\newcommand{\ent}[2]{[\![ #1, #2 ]\!]}



% Algèbre linéaire

\newcommand{\ev}{espace vectoriel}
\newcommand{\sev}{sous-espace vectoriel}
\newcommand{\kev}{$\K$-\ev}
\newcommand{\jus}{\underline{Justification :}}

\newcommand{\sta}{\mathrm{Stab}}
\newcommand{\card}[1]{\mathrm{card}(#1)}
\newcommand{\vect}[1]{\mathrm{Vect}\left(#1\right)}
\newcommand{\rg}[1]{\mathrm{rg}(#1)}
\newcommand{\mat}[2]{\mathrm{Mat}(#1 , #2)}

\newcommand{\im}{\mathop{\mathrm{Im}}\nolimits}
\renewcommand{\ker}{\mathop{\mathrm{Ker}}\nolimits}

\newcommand{\Gt}{G_{\mathrm{tor}}}



% Fonctions hyperboliques

\renewcommand{\th}{\mathop{\mathrm{th}}\nolimits}
\newcommand{\ch}{\mathop{\mathrm{ch}}\nolimits}
\newcommand{\sh}{\mathop{\mathrm{sh}}\nolimits}

\newcommand{\argsh}{\mathop{\mathrm{Argsh}}\nolimits}
\newcommand{\argch}{\mathop{\mathrm{Argch}}\nolimits}
\newcommand{\argth}{\mathop{\mathrm{Argth}}\nolimits}



% Raccourcis


\newcommand{\ssi}{si et seulement si}

\newcommand{\equi}{les assertions suivantes sont équivalentes :}

\newcommand{\pgcd}{\mathrm{PGCD}} 

\newcommand{\con}{$\text{\ding{54}}$ Contradiction.} 

\newcommand{\dis}{\displaystyle}
\newcommand{\dx}{\mathrm{d} x}
\newcommand{\dt}{\mathrm{d} t}

\newenvironment{ls}{\begin{list}{$\to$}{}}{\end{list}}
\newenvironment{lls}{\begin{list}{$\ast$}{}}{\end{list}}
\newenvironment{llls}{\begin{list}{$\cdot)$}{}}{\end{list}}


% Topologie

\newcommand{\bfe}{\overline{B}}

\newcommand{\nor}{\Vert \ \Vert}
\newcommand{\norm}[1]{\left\Vert #1 \right\Vert}
%\newcommand{\nt}[1]{|\!|\!|  #1 |\!|\!| }%
\newcommand{\nt}[1]{\llbracket  #1 \rrbracket }

\newcommand{\ov}[1]{\overline{#1}}


\newcommand{\inte}[1]{\mathring{ #1}}
\newcommand{\ad}[1]{\overline{ #1}}
\newcommand{\fr}[1]{\text{Fr}(#1) }
\newcommand{\lims}[1]{\overset{n \rightarrow + \infty}{\underset{#1}{\longrightarrow}}}
%\newcommand{\limf}[2]{\overset{x \rightarrow #1}{\underset{\nor_{#2}}{\longrightarrow}}}%
\newcommand{\limsu}{\overset{n \rightarrow + \infty}{\longrightarrow}}
\newcommand{\limf}[1]{\overset{x \rightarrow #1}{\longrightarrow}}





\newcommand{\mn}[1]{\mathcal{M}_n(#1)}
\newcommand{\mnk}{\mn{\K}}
\newcommand{\gl}[1]{\text{GL}(#1)}
\newcommand{\gle}{\gl{E}}
\newcommand{\gln}[1]{\text{GL}_n(#1)}
\newcommand{\glnk}{\gln{\K}}
\newcommand{\on}[1]{\text{O}(#1)}
\newcommand{\onn}{\on{n}}
\newcommand{\son}[1]{\text{SO}(#1)}
\newcommand{\sonn}{\son{n}}
\renewcommand{\mat}[2]{\underset{#1}{\text{Mat}}#2}


\newcommand{\tra}[1]{\,{\vphantom{#1}}^{t}\!{#1}}

\newcommand{\lir}[2]{\substack{#1 \to #2}}
\newcommand{\li}[2]{\lim\limits_{\substack{#1 \to #2}}}
\newcommand{\lix}[1]{\lim\limits_{\substack{#1}}}
\newcommand{\lis}{\li{n}{+\infty}}

\newcommand{\un}{{(u_n)}_{n \in \N} }
\newcommand{\uu}[1]{{(u_{#1})}_{n \in \N}}
\newcommand{\pp}[1]{{\left( #1\right)}_{n \in \N}}

\newcommand{\fn}{{(f_n)}_{n \in \N} }
\newcommand{\sfn}{\dis \sum f_n }


\newcommand{\f}{\forall \ }
\newcommand{\e}{\exists \ }
\newcommand{\va}[1]{\left\vert #1 \right\vert }

\renewcommand{\o}[1]{\mathrm{o}\left(#1\right)}

     
%________________________________________________________________________%

% Théorème, définition ...

\theoremstyle{definition}   

% avec numerotation

\newtheorem{defi}{Définition}[section]
\newtheorem{depro}[defi]{Définition-proposition}
\newtheorem{theo}[defi]{Théorème}
\newtheorem{prop}[defi]{Proposition}
\newtheorem{lem}[defi]{Lemme}
\newtheorem{coro}[defi]{Corollaire}
\newtheorem{ax}[defi]{Axiome}

%sans numerotation

\newtheorem*{defi*}{Définition}
\newtheorem*{depro*}{Définition-proposition}
\newtheorem*{theo*}{Théorème} 
\newtheorem*{prop*}{Proposition}
\newtheorem*{lem*}{Lemme} 
\newtheorem*{coro*}{Corollaire} 
\newtheorem*{ax*}{Axiome}
                
\theoremstyle{remark}

% avec numerotation

\newtheorem{rem}{Remarque}
\newtheorem{nota}{Notation}

%sans numerotation

\newtheorem*{rem*}{Remarque}
\newtheorem*{nota*}{Notation} 


\newcommand*{\quo}[2]%
{\ensuremath{%
    #1/\!\raisebox{-.65ex}{\ensuremath{\mathcal{#2}}}}}

%________________________________________________________________________%

% Définition personnalisée de la numerotation des titres, sous-titres ...

%\renewcommand{\thesection}{\Roman{section}.}
%\renewcommand{\thesubsection}{\alph{subsection}.}
%\renewcommand{\thesubsubsection}{\alph{subsubsection})}

%________________________________________________________________________%

% Structure

% Titres personnalisés

\newcommand{\partie}[1]{\textcolor{red}{\section{#1}}}
\newcommand{\titre}[1]{\textcolor{orange}{\subsection{#1}}}

% Commande exercice et exemple

\newcommand{\exems}{{\textbf{$\text{\ding{46}}$ Exemples : \ }}}

\newcommand{\exem}{{\textbf{$\text{\ding{46}}$ Exemple : \ }}}

\newcounter{nexo}
\setcounter{nexo}{0}        
\newcommand{\exo}{
  \stepcounter{nexo}        
  {\textbf{Exercice \arabic{nexo}} \\ }}
  
\newcommand{\appli}{{\textbf{$\text{\ding{46}}$ Application : \ }}}  

% Application

\newcommand{\app}[5]{#1:\left\{\begin{array}{rcl}
#2 & \longrightarrow & #3 \\
#4 & \longmapsto & #5  \\
\end{array}\right.
}

% Encadrement

\newcommand{\enc}[1]{\noindent\fbox{\parbox{\linewidth\fboxrule\fboxsep}{#1}}}
  
%________________________________________________________________________%

% Style de numérotation équation
  
\renewcommand{\theequation}{\arabic{equation}}
  
%________________________________________________________________________%

% Listes théorèmes, définitions
  
\newlist{az}{enumerate}{3}
\setlist[az,1]{label=\fcolorbox{blue}{lightgray}{\alph*},font=\color{black}}
\setlist[az,2]{label=(\roman*)}
\setlist[az,3]{label=$\leadsto$}

%________________________________________________________________________%

% Listes théorèmes, définitions correspondant aux démonstrations  
  
\newlist{dem}{enumerate}{1}
\setlist[dem,1]{label=\textbf{\alph*.}}


%________________________________________________________________________%

% Listes exercices, exemples
  
\newlist{exe}{enumerate}{2}
\setlist[exe,1]{label=\textcircled{\arabic*} }
\setlist[exe,2]{label= \alph*)}

%________________________________________________________________________%

% Structure démonstration (existence et unicité)

\newcommand{\demo}[2]{\begin{enumerate}[label=\textbullet]
\item \textsc{Existence :} #1
\item \textsc{Unicité :} #2
\end{enumerate}}

%________________________________________________________________________%

% Liste avec point
  
\newlist{point}{enumerate}{1}
\setlist[point,1]{label=$\bullet$}

%________________________________________________________________________%

% Liste numérotée
    
\newlist{lnu}{enumerate}{1}
\setlist[lnu,1]{label=(\arabic*) }

% Liste avec étoile 

\newlist{etoile}{enumerate}{2}
\setlist[etoile,1]{label=$\star$}
\setlist[etoile,2]{label= -}

%________________________________________________________________________%



% Listes cas
  
\newenvironment{cas}[1][gray]
  {\begin{description}[font=\color{#1}]}
  {\end{description}}



%________________________________________________________________________%

%_______________________________________________________________________Début document_____________________________________________________________________%


 
\begin{document}

\title{Notes de cours de M305-Alg\`ebre 2}

%\maketitle
\makeatletter
\begin{titlepage}
    \vspace*{\fill}
    \begin{center}
      {\Huge \@title}\\[0.5cm]
%      {\Large \@author}\\[0.4cm]
      {\Large \@date}
    \end{center}
    \vspace*{\fill}
\end{titlepage}
\makeatother

\newpage

\tableofcontents

\newpage

%\chapter*{Algèbre 2}

\section{Rappels de th\'eorie des groupes}

{\bf\`A compl\'eter}

\section{Groupes ab\'eliens de type fini}

Le but de cette section est de donner une classification des groupes ab\'eliens. On va s'int\'eresser pour cela \`a une sous-cat\'egorie de groupes ab\'eliens: les groupes abliens de type fini. Dans toutes cette section la loi des groupes consid\'er\'es sera $+$ sauf mention contraire explicite.



\begin{rem}
Tout groupe ab\'elien $(G,+,0_G)$ peut s'identifier \`a un $\Z$-module en posant:
$$\forall (n,g)\in\Z\times G,\, n\cdot g:=\begin{cases}0_G & si\, n=0\\ ((n-1)\cdot g)+g & si\, n>0\\((n+1)\cdot g) + (-g) & si\, n<0\end{cases}$$
On v\'erifie ais\'ement que $\cdot$ est bien d\'efinie et v\'erifie:\\
$$\forall n\in \Z,\forall (h,g)\in G\times G,\, n\cdot(h+g)=(n\cdot g)+(n\cdot h)$$
\end{rem}

\begin{depro}
Soient $G$ un groupe abélien et $S$ une partie de $G$. Alors il existe un unique sous-groupe $H$ de $G$ qui contient $S$ minimal (pour l'inclusion). $H$ est appelé le \textit{sous-groupe de $G$ engendré par la partie $S$}, et noté $\langle S\rangle$.
\end{depro}

\begin{proof}
On note $\mathcal{E}$ l'ensemble des sous-groupes de $G$ contenant $S$. $\mathcal{E}$ est non vide (car $G \in\mathcal{E}$). Alors $\dis \bigcap_{H \in \mathcal{E}} H$ est un sous-groupe de $G$ contenant $S$ qui contenu dans tout autre élément de $\mathcal{E}$.
\end{proof}

\begin{rem}
Soit $G$ un groupe abélien, soit $S$ une partie de $G$. Alors $\langle S\rangle$ contient l'élément neutre, et stable par additivité et passage à l'inverse. Comme $G$ est abélien, $\langle S\rangle$ contient donc toutes les combinaisons linéaires d'éléments de $S$, i.e.:
$$\left\{\dis \sum_{i=1}^r n_is_i \mid (s_i)_{1 \leqslant i \leqslant r} \in S^r, (n_i)_{1 \leqslant i \leqslant r} \in \Z^r, r \in \N \right\}\subset\langle S\rangle$$
On constate qu'un tel ensemble est de plus un sous-groupe de $G$ et donc, par unicité, on a m\^eme:
$$\langle S\rangle=\left\{\dis \sum_{i=1}^r n_is_i \mid (s_i)_{1 \leqslant i \leqslant r} \in S^r, (n_i)_{1 \leqslant i \leqslant r} \in \Z^r, r \in \N \right\}$$
\end{rem}

\begin{defi}
Soit $G$ un groupe abélien, soit $S$ une partie de $G$. 
\begin{point}
\item On dit que $S$ est u\textit{ne partie génératrice} de $G$, ou encore que $S$ engendre $G$, si $G=\langle S\rangle$.
\item On dit que le \textit{groupe abélien} $G$ est de \textit{type fini} s'il admet une \textit{partie génératrice finie}. 
\end{point}
\end{defi}

Pour tout $r \in \N^*$, on note $e=(e_i)_{1 \leqslant i \leqslant r} \in \left(\Z^r\right)^r$ la base canonique de $\Z^r$ (chaque vecteur a pour i\up{ème} coordonnée $1$ et $0$ ailleurs).

\exem Un groupe $G$ abélien fini est de type fini (engendré par la partie $G$ finie). \\
Soit $r \in \N^*$. Alors $\Z^r$ est un groupe abélien de type fini (engendré par la partie $\{e_1, \ldots, e_r \}$i
\begin{lem}\label{morphisme-Zr-G}
Soit $G$ un groupe abélien, soit $r \in \N^*$, soit $(x_1, \ldots, x_r) \in G^r$. Alors il \textit{existe un unique morphisme} de groupes $f$ du groupe $\Z^r$ dans le groupe $G$ qui, pour tout $i \in \nt{1,r}$, \textit{envoie $e_i$ sur $x_i$ $(\star)$}, donné par $\app{f}{\Z^r}{G}{(a_i)_{1 \leqslant i \leqslant r}}{\dis \sum_{i=1}^ra_ix_i}$. \\ De plus, $\im f =\langle\{x_1, \ldots, x_r\}\rangle$.
\end{lem}

\begin{proof}
Un morphisme de groupes de $\Z^r$ dans $G$ vérifiant $(\star)$ est uniquement déterminé par l'image de la base canonique $e$ de par la décomposition unique d'un vecteur de $\Z^r$ dans cette base et la linéarité du morphisme.
\end{proof}

\begin{coro}\label{type-fini-et-Zr}
Un groupe abélien est de type fini si et seulement si il existe un morphisme de groupes \textit{surjectif} du groupe $\Z^r$ dans le groupe $G$ (où $r \in \N$).
\end{coro}

\begin{proof}
On conserve les notations du lemme~\ref{morphisme-Zr-G}. \\
Soit $G$ un groupe abélien de type fini. Alors $G$ possède une partie génératrice finie $S=\{x_1, \ldots, x_r \}$ (où $r \in \N$), d'où $\im f=\langle S\rangle=G$,
i.e. le morphisme de groupes $f$ de $\Z^r$ dans $G$ est surjectif. \\
Réciproquement, soit $g$ un morphisme de groupes surjectif de $\Z^r$ (où $r \in \N$) dans $G$ , alors $S=\{f(e_1), \ldots, f(e_r) \}$ est une partie génératrice de $G$ (car tout élément de $G$ possède un antécédent dans $\Z^r$ qui se décompose dans la base canonique $e$, donc son image, i.e. $g$, s'écrit comme combinaison linéaire d'éléments de $S$). 
\end{proof}

\begin{prop}\label{image-type-fini}
Soient $G$ et $H$ deux groupes abéliens. Soit $f$ un morphisme de groupes de $G$ dans $H$. On suppose que le groupe abélien $G$ est de type fini. Alors le groupe abélien $\im f$ est de type fini.
\end{prop}

\begin{proof}
Soit $S$ une partie génératrice de $G$. Alors $f(S)$ est une partie génératrice de $\im f$.
\end{proof}

\begin{coro}
Soit $G$ un groupe ab\'elien de type fini et $H$ un sous-groupe de $G$. Alors le quotient $G/H$ est de type fini.
\end{coro}

\begin{proof}
Il suffit d'appliquer la proposition~\ref{image-type-fini} \`a la sujection canonique $\pi:G\to G/H$.
\end{proof}

\begin{prop}\label{g-type-fini-par-morphisme}
Soient $G$ et $H$ deux groupes abéliens. Soit $f$ un morphisme de groupes de $G$ dans $H$. On suppose que les groupes abéliens $\im f$ et $\ker f$ sont de \textit{type fini}. Alors $G$ est un groupe abélien de type fini.
\end{prop}

\begin{proof}
Par hypothèse, il existe $(x_1, \ldots, x_r) \in G^r$ tel que la partie finie $\{f(x_1), \ldots, f(x_r) \}$ engendre $\im f$ et il existe $(y_1, \ldots, y_r) \in (\ker f)^s$ tel que la partie finie $\{y_1, \ldots, y_s \}$ engendre $\ker f$. Soit $g \in G$. Alors il existe $(n_1, \ldots, n_r) \in \Z^r$ tel que $f(g)=\dis \sum_{i=1}^r n_if(x_i)=f\left(\sum_{i=1}^r n_ix_i\right)$. On conclut en décomposant l'élément $g-\dis \sum_{i=1}^r n_ix_i\in \ker f$ selon la famille $\{y_1, \ldots, y_s\}$.
\end{proof}

\begin{prop}
Soit $G$ un groupe abélien de type fini. Soit $H$ un sous-groupe de $G$. Alors le groupe abélien $H$ est de type fini.
\end{prop}

\begin{proof}
\begin{ls}
\item Supposons le groupe $G$ monogène. Si $G$ est cyclique, on sait alors que $H$ est cyclique. Sinon, on se donne un générateur $g$ de $G$ (qui est donc d'ordre infini). On sait que l'application $\app{\varphi}{\Z}{G}{n}{ng}$ est un isomorphisme de groupes. $H$ est isomorphe au sous-groupe $\varphi^{-1}(H)$ de $\Z$ qui s'écrit donc $\varphi^{-1}(H)=m\Z$ (où $m \in \N$), i.e. $H=\varphi(m\Z)=\{ng \mid n \in m\Z \}=\langle mg\rangle$ i.e. $H$ est monogène. 
\item Dans le cas contraire, on considère un morphisme de groupes $f:\Z^r\to G$ surjectif, où $r \in \N^*$ désigne le cardinal de la partie génératrice de $G$ (cf preuve du corollaire~\ref{type-fini-et-Zr}). Montrons par récurrence sur $r \in \N^*$ que $H$ est de type fini:
\begin{lls}
	\item\underline{Initialisation $r=1$:} C'est le cas o\`u $G$ est monog\`ene et a donc d\'ej\`a été traité dans le premier point de cette preuve.
	\item\underline{H\'er\'edit\'e:} Supposons le résultat vrai pour $r-1$ ($r \in \N^*$). Considèrons le groupe $\Z^{r-1}$ comme sous-groupe de $\Z^r$ (via l'injection de $\Z^{r-1} \times \{0\}$ dans $\Z^{r}$). Par hypothèse de récurrence, tout sous-groupe de $K=f(\Z^{r-1})$ est de type fini. Montrons que $G/K$ est monogène : considérons le morphisme de groupes $\app{\psi}{\Z}{G/K}{a}{\overline{f(0, \ldots, 0,a)}}$. 
Soit $\overline{g} \in G/K$. Par surjectivité de $f$, il existe $(a_1, \ldots, a_r) \in \Z^r$ tel que $g=f(a_1, \ldots, a_r)$. Donc $g=\underbrace{f(a_1, \ldots, a_{r-1},0)}_{\in K}+f(0, \ldots, 0, a_r)$, i.e. $\overline{g}=\overline{f(0, \ldots, 0,a_r)}=\psi(a_r)$. Ainsi, $\psi$ est surjectif, et donc $G/K$ est monog\`ene d'apr\`es ce qui a \'et\'e montr\'e au premier point de cette preuve.\\
Soit $H$ un sous-groupe de $G$.
Considèrons $\pi_H:H\to G/K$ la restriction \`a $H$ de la surjection canonique. $\im \pi_H$ est un sous-groupe de $G/K$ monogène, donc $\im \pi_H$ est monogène. \\
$\ker \pi_H=K \cap H \subset K$ est de type fini par hypothèse de récurrence. D'après la proposition~\ref{g-type-fini-par-morphisme}, on a donc que $H$ est de type fini.
\end{lls}
\end{ls}
\end{proof}

\begin{defi}\hspace{.1em}
\begin{ls}
\item Soit $G$ un groupe quelconque:
\begin{lls}
\item Soit $g \in G$. On dit que l'élément $g$ \textit{est de torsion} si $g$ est d'ordre fini.
\item On dit que le groupe $G$ est \textit{de torsion} si tout élément de $G$ est de torsion.
\item On dit que le groupe $G$ est \textit{sans torsion} si tout élément de $G$ non nul est d'ordre infini.
\end{lls}

\item Si $G$ est de plus abélien, on dit que $G$ est \textit{libre de type fini} s'il existe $r \in \N$ tel que $G$ soit isomorphe à $\Z^r$.\\
Dans ce cas, on a:
\begin{lls}
\item Soit $(x_1, \ldots, x_r) \in G^r$.  On dit que les éléments $x_1, \ldots, x_r$ forment une famille libre, ou encore sont linéairement indépendants, si:
$$\forall (n_1, \ldots, n_r) \in \Z^r,\, \left(\dis \sum_{i=1}^r n_ix_i=0\implies\forall i \in \nt{1,r}, n_i=0\right)$$
\item On dit que les éléments $x_1, \ldots, x_r$ forment une base de $G$ si ils forment une famille libre et génératrice de $G$.
\end{lls}
\end{ls}
\end{defi}

\begin{depro}\label{def-groupe-torsion}
Soit $G$ un groupe abélien.
\begin{point}
\item Il existe un unique sous-groupe $H$ de $G$ de torsion qui contient tout sous-groupe de $G$ de torsion. $H$ est appelé le sous-groupe de torsion de $G$, et est noté $\Gt$.
\item Le groupe quotient $G/\Gt$ est sans torsion.
\item Soit $(x_1, \ldots, x_r)  \in G^r$. Alors la famille  $(x_1, \ldots, x_r)$ est libre si et seulement si le morphisme de groupes $f:\Z^r\to G$ du lemme~\ref{morphisme-Zr-G} est injectif.
\end{point}
\end{depro}

\begin{proof}\hspace{.1em}
\begin{point}
\item Le sous-ensemble $H$ de $G$ constitué de l'ensemble des éléments de torsion de $G$ est un sous-groupe de torsion de $G$ (l'élément nul est d'ordre $1$, la somme de deux éléments de torsion est d'ordre le ppcm des ordres, et l'ordre de l'inverse d'un élément de torsion est inchangé) et contient tout sous-groupe de torsion par construction.
\item Soit $\overline{g} \in G/\Gt$ de torsion : il existe $n \in \N^*$ tel que $n\overline{g}=\overline{ng}=\overline{0}$, i.e. $ng \in \Gt$ : il existe $m \in \N^*$ tel que $mng=0$ d'où $g$ est de torsion, i.e. $\overline{g}=\overline{0}$. Donc $G/\Gt$ est sans torsion.
\item $f$ est injectif si et seulement si $\ker f=\left\{(a_i)_{1 \leqslant i \leqslant r} \in \Z^r \mid \dis \sum_{i=1}^r a_ix_i=0\right\}=\{0_{\Z^r}\}$.
\end{point}
\end{proof}

\begin{coro}
Soit $G$ un groupe abélien de type fini. Alors $G$ est libre de type fini si et seulement si il admet une base.
\end{coro}

%\begin{proof}Soit $\{x_1, \ldots ,x_r\}$ une partie gnratrice de $G$. Alors, le groupe ablien $G$ est de type fini est libre si et seulement si $F$ est un morphism  e de groupes injectif ($F$ tant surjectif car $G$ de type fini, $F$ est alors un isomorphisme par cardinalit), soit si et seulement si (d'après la proposition prc  dente) $(x_1, \ldots, x_r)$ est libre, ie est une base (tant gnratrice de $G$).
%\end{proof}
%on ne peut pas utiliser ici la cardinalit\'e, car ce sont des espaces infinis, et on ne peut pas utiliser l'alg\`ebre lin\'eaire, car on n'a pas de r\'esultats sur les modules

\begin{proof}\hspace{.1em}\\
\underline{$\Rightarrow$} Supposons $G$ libre de type fini et soit $r\in\N$ tel que $f:\Z^r\to G$ soit un isomorphisme de groupes. Soit $(x_1, \ldots ,x_r)=(f(e_1),\ldots, f(e_r))$ les images de la base canonique $e=(e_i)_{1\leqslant i\leqslant r}$ de $\Z^r$ par $f$. Soit $z\in\Z^r$ et $(a_1,\ldots,a_r)\in\Z^r$ tel que $z=\dis\sum_{i=1}^r a_ie_i$. On a donc:
$$f(z)=f(\dis\sum_{i=1}^r a_i e_i)\stackrel{\textrm{$f$ morphisme}}{=}\dis\sum_{i=  1}^r a_if(e_i)=\dis\sum_{i=  1}^r a_ix_i$$
$f$ est donc de la forme du lemme~\ref{morphisme-Zr-G} et de par la proposition-d\'efinition~\ref{def-groupe-torsion}, on a que $(x_1, \ldots ,x_r)$ est libre. De plus par isomorphisme on a que $\{x_1, \ldots ,x_r\}$ est génératrice de $G$. $(x_1, \ldots ,x_r)$ est donc une base de $G$.\\
\underline{$\Leftarrow$} Soit $(x_1, \ldots ,x_r)$ une base de $G$ et soit $f$ le morphisme du lemme~\ref{morphisme-Zr-G} associ\'e. Alors d'apr\`es la proposition-d\'efinition~\ref{def-groupe-torsion}, $f$ est injectif et d'apr\`es la preuve du corollaire~\ref{type-fini-et-Zr}, $f$ est surjectif, i.e. $f$ est un isomorphisme de groupes.
\end{proof}

\begin{prop}
Soit $G$ un groupe abélien de type fini, alors le groupe de torsion $\Gt$ est fini.
\end{prop}

\begin{proof}
Soit $\{x_1, \ldots ,x_r\}$ une partie génératrice de $\Gt$ (licite car $\Gt$ est de type fini comme sous-groupe d'un groupe de type fini). Pour tout $i \in \nt{1,r}$, on note $n_i$ l'ordre de l'élément $x_i$. On considère le morphisme de groupes $\app{\varphi}{\Z/n_1\Z \times \cdots \times \Z/n_r\Z}{G}{(\overline{a_1}, \ldots, \overline{a_r})}{\dis \sum_{i=1}^r a_ix_i}$.\\
$\varphi$ est bien défini car pour tout $(\overline{a_1}, \ldots, \overline{a_r})=(\overline{b_1}, \ldots, \overline{b_r}) \in \Z/n_1\Z \times \cdots \times \Z/n_r\Z$, $(a_i-b_i)\in n_i\Z$, donc $n_i=ordre(x_i)\mid (a_i-b_i)$ et donc $\dis \sum_{i=1}^r \underbrace{(a_i-b_i)x_i}_{=0}=0$.\\
De plus $\varphi$ est surjectif par construction  donc $\vert G \vert \leqslant \dis \prod_{i=1}^r n_i \in \N^*$ i.e. $G$ est fini.
\end{proof}

\end{document}

